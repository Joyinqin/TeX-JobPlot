
\begin{center}
	\Large{\textbf{摘\ 要}}
\end{center}

随着大规模电子商务行业中数据传输量的增加,管理如此大量的请求成为一个关键问题。通过在分布式服务器系统上应用任务调度来处理请求是一种有效的方法。但是,当服务器节点在短时间内处理极大量的请求时,过高负载量是不可避免的。没有有效的高负载监视方法,很难高效的对后端服务器集群进行监控和管理。据我们所知,监控高负载异常节点的现有方法既不灵活也不直观,并且不能检测服务器节点异常,例如定位客户端发送的不合理请求。在本文中,我们提出了一种基于真实数据集的可视化分析的作业调度监控方法,该方法允许监控人员了解该区域中运行节点的状态,并通过各种视图组件观察导致高负载服务器的可疑请求。这种思路为服务器端任务调度集群监测提供了一种全新的方法,并且经过测试显示是可行且有效的。
	
	\textbf{关键词:数据可视化\ 任务调度\ 异常检测}
	%插入空格:\quad或者\ .


%本页不显示页码
\pagenumbering{Roman}
\newpage
%本页不显示页码


\begin{center}
	\large{\textbf{Abstract}}
\end{center}

With the increasing amount of data transmission in large-scale e-commerce industry, managing such an enormous amount of requests simultaneously becomes a key problem. It is an effective method to handle the requests by applying task scheduling on the distributed server system. However, high load on the servers is inevitable in scheduling when requests of a server node process extreme large amount of tasks in a short period of time. It is difficult to maintain load balance without an effective high-load surveillance approach. To best of our knowledge, existing methods of monitoring high-load abnormal nodes are neither flexible or intuitive and are not capable of detecting server node anomalies such as positioning unreasonable requests sent by the clients. In this paper, we propose a job scheduling monitoring method based on visual analysis from a real dataset, which allows the monitoring personnel to know the status of running nodes in the area and observe the suspected requests causing a high load of servers via various view components we inject into the system.

\textbf{Keywords: Task scheduling,\ visual analysis,\ abnormal detection}



\newpage
