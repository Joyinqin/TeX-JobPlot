\section{第一章\quad 绪\ 论}
\subsection{研究工作的背景与意义}
现如今网络电子商务项目正在蓬勃发展,人们在享受着足不出户购物和信息浏览,但是往往当使用者的数目变得足够多时,安全隐患往往随之到来。尽管各大互联网企业采取了一些应对措施,比如将用户节点置于类比虚拟机的容器中,或者使用算法对用户请求进行调度处理,使指令被分散在不同的服务器上进行处理,但如果控制这台用户节点的服务器因为某些原因出现超负荷运转,甚至宕机的话,目前整个整个相关领域是没有很好的监视系统来全程监视负责处理用户请求的服务器的运行状态的。比如全球先进的超级计算机研究所——美国德克萨斯州高级计算中心(TACC)现在存在检测正在运行服务器的监视器,但是功能和用户操作流程十分不人性化:他们必须通过具有丰富经验、工作年限很长的后台管理者,通过报错信息得到运行异常的服务器编号,再去监测系统手动输入,通过查看后端返回的一系列运行参数,通过经验判断这台服务器的出错原因从而采取补救措施。类似这样的做法,缺点十分明显,比如在整个操作的过程中,操作流程十分僵硬,充斥着很多人为感知和人为判断,使得操作效率的低下;或者在某一时刻,如果因为在某一地区的用户普遍都出现了系统故障导致在短时间内出现了大面积的服务器运行异常,在如此庞大的工作量之下,工作人员的补救效率在不先进的系统下是很迟钝的,这种中间环节的纰漏就会导致解决过程的滞后,从而引发连带问题。

\subsubsection{title}
\paragraph{title}
\subparagraph{title}