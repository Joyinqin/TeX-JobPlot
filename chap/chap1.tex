\setcounter{page}{1}
\pagenumbering{arabic}
\chapter{绪\ 论}
\section{研究工作的背景与意义}
现如今网络电子商务项目正在蓬勃发展,人们在享受着足不出户购物和信息浏览,但是往往当使用者的数目变得足够多时,安全隐患往往随之到来。尽管各大互联网企业采取了一些应对措施,比如将用户节点置于类比虚拟机的容器中,或者使用算法对用户请求进行调度处理,使指令被分散在不同的服务器上进行处理,但如果控制这台用户节点的服务器因为某些原因出现超负荷运转,甚至宕机的话,目前整个整个相关领域是没有很好的监视系统来全程监视负责处理用户请求的服务器的运行状态的。比如全球先进的超级计算机研究所——美国德克萨斯州高级计算中心(TACC)现在存在检测正在运行服务器的监视器,但是功能和用户操作流程十分不人性化:他们必须通过具有丰富经验、工作年限很长的后台管理者,通过报错信息得到运行异常的服务器编号,再去监测系统手动输入,通过查看后端返回的一系列运行参数,通过经验判断这台服务器的出错原因从而采取补救措施。类似这样的做法,缺点十分明显,比如在整个操作的过程中,操作流程十分僵硬,充斥着很多人为感知和人为判断,使得操作效率的低下;或者在某一时刻,如果因为在某一地区的用户普遍都出现了系统故障导致在短时间内出现了大面积的服务器运行异常,在如此庞大的工作量之下,工作人员的补救效率在不先进的系统下是很迟钝的,这种中间环节的纰漏就会导致解决过程的滞后,从而引发连带问题。
\section{本文的主要贡献与创新}
本论文的贡献主要是,通过数据可视化手段,实现了对服务器工作状态的全过程实时监测,克服了以前人为操作的种种不利,在功能上也进行了扩充,可以实现的具体功能和创新点如下:
\begin{itemize}
	\item 及时而直观地了解到时间节点下,正常和异常服务器的运行状态参数(CPU利用率、memory利用率、disk利用率等)和服务器本身的各项硬件指标(CPU数目、memory大小等),通过时间和服务器节点实时切换;
	
	\item 通过服务器所带的任务(task)、作业(job)和实例(instance)信息,分析出导致服务器运行异常的原因是由什么任务,或者哪一台用户的异常操作而引起的,在可视化视图中分析异常原因;
	
	\item 直观地观察到服务器任务(task)、作业(job)和实例(instance)之间的包含关系和数目大小,及时得到主要占用服务器资源的作业信息,做出相应调整;
	
	\item 通过热力图刷新迅速掌握某服务器节点在过去一段时间的工作状态,帮助工作人员寻找规律,辅助判断异常节点的出错动机;
	
	\item 了解到在整个有记录的时间线上所有服务器节点的运行状态和某一时刻下的全平台的服务器异常数目,定位异常高峰,联动其他组件有针对性地确定需要观察的时刻。

\end{itemize}
	
总而言之,该方法的创新点就是在传统的监测系统的基础上,通过增加可视化元素,使操作过程变得简洁流畅;在此思路的基础上,增加了各种功能,来帮助监测人员合理地定位异常,通过可视手段分析异常原因,定位故障源,从而有效地实时控制服务器的运行状态,为庞大的商业系统的正常运行保驾护航。

\section{本论文的结构安排}
本文的章节结构安排如下:

第一章:绪论部分,简述研究背景和现阶段研究进展,分析需求从而得出需求;

第二章:详细阐述本文的相关研究、研究意义、研究手段、简述数据可视化方法和研究数据集等研究基础;

第三章:分析系统设计思路,按模块化阐述系统的各部分功能,通过操作视图来具体演示系统的工作流程从而实例化分析,并且加入了测试用例来检验我们想法的正确性和系统的可行性;

第四章:全文的总结和归纳,分析系统仍然存在的缺陷,以及对未来工作的展望;

第五章:参考文献

\newpage
