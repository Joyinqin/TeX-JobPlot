\section{第二章\quad 项目意义及背景介绍}
\subsection{项目意义}
对电子商务任务调度模式下的异常监测的可视化实现,对此类控制平台提出了一种新的思路和方法,也是数据可视化在交叉领域的全新进展。
\subsubsection{电商集群的规模化}
2018年7月10日,2018中国互联网大会发布了新一版的《中国互联网发展报告(2018)》。报告中指出,2017年,中国电子商务交易服务营收规模为5027亿元,首次突破5000亿大关。2017年第三方互联网支付也达到143.26万亿,网络购物市场交易规模达5.33万亿元,而网络零售的市场交易规模为7.18万亿。中国网上支付用户规模达5.31亿人,其中手机支付用户就达5.27亿人,较2016年底增加5783万人,年增长率为12.3\%,规模增长迅速。正如网民在现实生活中体会到的,电子商务交易已经成为我们购买日常用品的首选手段,在中国庞大的网民基数和中国网络技术飞速发展的双重影响下,这一数字在以后还会以大增幅增长。根据商务部统计,2020年预计中国网络零售市场规模为9.6万亿,是2012年的10倍之多。电子商务在深度影响着国民生活的同时,也掌握着货架经济的命脉,倘若电商集群由于某种原因崩溃的话,给整个国家带来的影响即将是灾难性的。

中国电子商务的龙头企业——阿里巴巴网络技术有限公司(以下简称阿里巴巴),在国内的电子交易平台里扮演着举足轻重的地位。根据2018年阿里巴巴公布的财年财报显示:2018全年,阿里巴巴营收 2502.66 亿元人民币(约 398.98 亿美元),同比增长 58\%,核心电商业务收入 2140.20 亿元人民币,同比增长 60\%,均创下IPO(Initial Public Offerings,指股份公司首次向社会公众公开招股的发行方式,简称IPO)以来年度最高增幅,2018 财年净利润为 832.14 亿元人民币(约 132.66 亿美元)。如此庞大的交易额和成交额使专家和研究人员不自主地将目光放在其身上。2018年阿里巴巴在Github上公布了一组数据,该数据刻画了阿里巴巴在8天地范围内4000台服务器的运行状态的数据,而这组数据也将成为本项目的研究数据集,在本章第三部分,将会着重对该数据进行详细介绍。

\subsubsection{故障问题及手段}
电子商务平台的实现其实并不是一个不能达到的要求,但是任何系统架构在达到一定规模之后,大大小小的问题往往会接踵而至,而这也是一个企业生存的关键。2018年8月1日,阿里巴巴旗下的淘宝网交易平台,淘宝服务器出现大范围的故障,全国多地网友在微博反馈称自己的淘宝崩溃无法查看订单,淘宝App、PC版网页均出现“网络竟然崩溃了”的提示,即使切换网络和重启手机也无效,在长达数小时的等待之后,该漏洞得到了修复。具体的原因,阿里巴巴并没有给出明确的回复,之后这件事也就不了了之。然而这已经不是阿里巴巴遇到的第一次服务器崩溃事件,每隔数月就会时不时的发生类似的服务器故障。由此可见,即使是如此宏大的电子商务企业也会因为后端服务器的宕机事件,造成企业形象的不良影响的同时也造成了利润的亏损。换句话说,如果能快速发现故障,找出导致该进程异常的原因,再利用分支等手段同时进行维护,结果将会截然不同。而本系统的设计初衷就是以阻止此类问题的发生而提出的。
\subsubsection{设计优势}
本系统设计将基于可视分析,在对服务器异常进行图形化展示的同时,还拥有异常定位、时序性监测等特点,对出现异常信号的服务器节点进行实时监测,并直观的展示造成该异常的任务,从而达到快速、准确的定位故障的目的。
\subsection{项目背景及相关理论}
\subsubsection{研究现状}
{面对异常检测问题}\cite{article2},Xiaowei Qin等人提出了一种{面向对象的检测框架}\cite{article3},它具有两步聚类,称为沙漏聚类。两个参数,关键质量指标和因果参数,通过结合自组织映射(SOM)和k-medoids的混合算法,将它们聚类成不同的类型。{Pei Yang等人}\cite{article4}。建议使用生成的拮抗网络(GAN)来检测异常。{Daojing He等人}\cite{article6},介绍使用软件定义网络(SDN)检测流量异常的优势。A.R.Jakhale\cite{article10}使用{数据挖掘}\cite{article7}\cite{article8}\cite{article9}技术,利用滑动窗口模型和聚类技术检查网络流的异常数据包。 {Alireza Tajary}\cite{article11}等,提出了一种吞吐量感知的瞬态故障检测方法,它利用了多核服务器处理器的特性。

为了识别大型,动态和异构数据中的异常\cite{article14},Nan Cao等,介绍一种视觉互动\cite{article20}\cite{article21}\cite{article22}系统和框架,称为Voila。该系统主要实现在线监测和与用户的互动。{Y.B. Luo等人}\cite{article23},提出了一种基于流量限制可穿透能见度图(FL-LPVG)的异常检测方法。该方法基于网络流序构建复杂网络,挖掘相关图的结构行为模式,提取网络流特征序列,利用LPG将统计特征序列转换为关联图,通过数据挖掘和信息检测异常流量基于熵的理论技术。其优点是该方法大大简化了异常检测过程,有效降低了高维数据的维数。但是为了提高这个系统的效率,我们必须从大量数据中完全挖掘行为特征。因此,它肯定会带来如何处理大数据以及如何提取有效信息的挑战。

{Josef Kittler 等人}\cite{article27},在解决异常检测问题时引入{域异常的概念}\cite{article24}\cite{article25}。异常有许多方面,每个域都是异常的许多方面之一。在此基础上,他们参考贝叶斯概率推理设备,并提出统一的{异常检测框架}\cite{article26},以识别和区分每个领域的异常。该设备通过定义各种异常属性来提供域异常事件的分类。该框架的创新特点是它暴露了异常的多方面性质,并且可以识别可能导致异常事件的各种原因,以及相应的检测机制。

\subsubsection{数据可视化}
本项目的特征是基于数据可视化的电商平台集群管理。数据可视化是由于视觉给人类带来的感知是最直观最有效率的一种获取信息的手段。目前国内的数据可视化行业也在蓬勃发展,涌现除了不少有能力的人才和惊艳的科技成果,下面将用一小部分篇幅来介绍一下这门领域的相关特点及其主要用途。

从生物学的角度来考虑,人类所有器官能接收到信息的80\%都来自于视觉,在大数据时代下,对信息的表达则就显得尤为关键,而数据可视化也就承担起了这一重要任务,充当着数据和人类之间的关键载体。顾名思义,数据可视化是关于数据视觉表现形式的科学技术研究。其中,这种数据的视觉表现形式被定义为,一种以某种概要形式抽提出来的信息,包括相应信息单位的各种属性和变量。它是一个处于不断演变之中的概念,其边界在不断地扩大。主要指的是技术上较为高级的技术方法,而这些技术方法允许利用图形、图像处理、计算机视觉以及用户界面,通过表达、建模以及对立体、表面、属性以及动画的显示,对数据加以可视化解释。
\subsubsection{电子商务分布式任务调度}
\subsubsection{数据集分析}
\subsection{关键问题}
\subsubsection{数据处理}
\subsubsection{集群表示及定位故障源}
\subsubsection{故障域算法}
\subsubsection{时序性及空间性}