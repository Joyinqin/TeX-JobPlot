\chapter{第三章\quad 系统设计与实现}
\section{项目意义}
\subsection{主要任务}

为了解决现有任务调度控制方法的不人性化和不直观以及拓展更多功能的需求,故提出本项目,旨在为解决以上问题而更近一步。具体任务列举如下:

\begin{itemize}
	\item 通过可视化分析的方法及时地了解到某时间节点下,正常和异常服务器的运行状态参数(CPU利用率、memory利用率、disk利用率等)和服务器本身的各项硬件指标(CPU数目、memory大小等),并且可以进行实时切换的功能;
	\item 通过调度信息(task、job和instance),由图形展示出导致异常的原因归咎于哪个具体task或job,由此得到用户操作对后端服务器的影响;
	\item 直观看到服务器任务(task)、作业(job)和实例(instance)之间的包含关系以及数目大小,得到主要占用服务器资源的作业信息,做出相应调整;
	\item 通过热力图刷新迅速掌握某服务器节点在过去一段时间的工作状态,帮助工作人员寻找规律,辅助其他模块判断异常节点的出错动机;
	\item 了解到在某一时间点下拥有记录的所有服务器节点的运行状态和该时间下运行某任务的服务器异常数目,定位异常高峰,并由此有针对性地确定需要观察的时刻。
\end{itemize}

本文拟根据以上需求设计并实现相应的系统模块,设计和实现过程在本章的余下章节将展开详细说明。

%\newpage
\subsection{关键问题的解决方法}

{\textbf{3.1.2.1\quad 数据处理}}

上文中提到,处理、读取并查找庞大量级的数据文件成为了一大难题。出于对本项目模型性和实时性的考虑,经过团队的讨论最终决定选择八天中第二天的数据作为数据集模拟数据流进行研究。然而问题是经过分割后的数据文件的体积仍然超过了研究地点所有独立设备的内存极限,解决方法在于python对数据处理方法的改变:如算法\ref{alg-1}所示,改变以往的处理中小型数据的思路和方法,将庞大的数据集每次按需读入设备内存进行处理,故得出两种可行的方法,python csv中的按行读取方法和pandas的块级读取的方法,即每次根据电脑内存大小只读取原数据文件的一小部分,等这一部分被处理和计算结束后再去重复操作处理其他部分的数据。这样便实现了任意大小的文件都可以被任意设备进行处理和计算的方法。

\begin{algorithm}
	\caption{Huge data file reade}
	\label{alg-1} 
	\textbf{Input}: data\_path
	
	1:\textbf{initialize}:reader=csvReader(data\_path)
	
	2:\textbf{for} row  \textbf{in} reader do
	
	3:\qquad filter by Conditions
	
	4:\qquad write(row)
	
	5:\textbf{end for}

\end{algorithm}

{\textbf{3.1.2.2\quad 集群表示及定位故障源}}

为了准确而简洁地表示每一个任务所持有的job、task与instance参数,以及如何将三者与具体被调度地服务器节点联系起来,经过讨论拟采用以下方案:采用可是分析展示方法中的tree结构,可以清晰明了地展示三者之间地层次关系。与此同时,在叶子节点的节点处与相应处理服务器绑定,反映正在运行此instance的服务器所处的运行状态。通过以上方法,就完美解决了任务调度信息的层次关系和不同数据文件之间的关联。

{\textbf{3.1.2.3\quad 故障域算法}}

数据集本身有许多参数来说明后端服务器节点的状态。 控制阈值是确定异常域的有效方法,取决于三个有价值的领域:CPU利用率,内存利用率和磁盘利用率。 通过为这三个参数分配权重,系统可以确定节点在处理任务实例时是否处于高负荷。 算法二描述了在我们的研究中识别异常的方法:

$$Value_{hl}=x\cdot\frac{S_{\beta +\gamma}}{S}+y\cdot\frac{S_{\alpha+\gamma}}{S}+z\cdot\frac{S_{\alpha+\beta}}{S}$$

在上面的算法中,$\alpha$,$\beta$和$\gamma$分别代表CPU,内存,磁盘的阈值。 我们使用概率统计学的思想来设置适当的阈值,以保持异常高负荷节点的比例接近10%。 可以像这样计算高负荷的负荷值。

{\textbf{3.1.2.4\quad 时序性及空间性}}
为了将数据可视化与节点状态的时序性结合起来,我们拟通过日历图的热力排布来展示一个服务器节点在过去一段时间内的运行状态。通过像素块颜色的深浅,直观反映某节点的健康程度。

由于缺少节点的地理位置信息,所以不能将集群的地理分布在地图上展示,前期拟定的三维数值坐标展示将只用树状结构叶子节点的热力色块来展示。因为使用者并不用清楚的知道此时此刻服务器的各种利用率参数,只需要知道该节点健康与否即可。